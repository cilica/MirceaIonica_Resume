% resume.tex
% vim:set ft=tex spell:

\documentclass[10pt,letterpaper]{article}
\usepackage[letterpaper,margin=0.75in]{geometry}
\usepackage[utf8]{inputenc}
\usepackage{mdwlist}
\usepackage[T1]{fontenc}
\usepackage{textcomp}
\usepackage{graphicx}
\usepackage{tgpagella}
\usepackage{hyperref}

\pagestyle{empty}
\setlength{\tabcolsep}{0em}

\graphicspath{ {./images/} }

% indentsection style, used for sections that aren't already in lists
% that need indentation to the level of all text in the document
\newenvironment{indentsection}[1]%
{\begin{list}{}%
	{\setlength{\leftmargin}{#1}}%
	\item[]%
}
{\end{list}}

% opposite of above; bump a section back toward the left margin
\newenvironment{unindentsection}[1]%
{\begin{list}{}%
	{\setlength{\leftmargin}{-0.5#1}}%
	\item[]%
}
{\end{list}}

% format two pieces of text, one left aligned and one right aligned
\newcommand{\headerrow}[2]
{\begin{tabular*}{\linewidth}{l@{\extracolsep{\fill}}r}
	#1 &
	#2 \\
\end{tabular*}}

% make "C++" look pretty when used in text by touching up the plus signs
\newcommand{\CPP}
{C\nolinebreak[4]\hspace{-.05em}\raisebox{.22ex}{\footnotesize\bf ++}}

% make "C#" look pretty when used in text by touching up the plus signs
\newcommand{\Csharp}
{C\nolinebreak[4]\hspace{-.05em}\raisebox{.22ex}{\footnotesize\bf \#}}

% and the actual content starts here
\begin{document}

\begin{center}
{\LARGE \textbf{Mircea Ionica}}

Sibiu\ \ \textbullet\ \ Romania
\\
0040.xxxxxx.xxx\ \ \textbullet
\ \ \url{https://www.linkedin.com/in/mircea-ionica/}
\end{center}

%\includegraphics[scale=0.275]{tags}

\hrule
\vspace{-0.4em}
\subsection*{Experience}

\begin{itemize}
	\parskip=0.1em

  \item
	\headerrow
		{\textbf{Freelancer}}
		{\textbf{Sibiu, Romania}}
	\\
	\headerrow
		{\emph{Expert Software Developer, Software Architect, Agile Team Leader}}
		{\emph{2016 -- present}}
	\begin{itemize*}
		\item Coordinating a cross-functional Agile Team
		\item Involved in architectural decisions and decompositions for ADAS related SW (Classic AUTOSAR)
		\item Design, implement and test Classic AUTOSAR DLT BSW module (C)
		\item Design, implement and test Adaptive AUTOSAR Applications (\CPP)
		\item Design, implement and test Software Components - SWC (C)
	\end{itemize*}


  \item
	\headerrow
		{\textbf{3Soft Automotive SRL}}
		{\textbf{Sibiu, Romania}}
	\\
	\headerrow
		{\emph{Senior Software Developer}}
		{\emph{2015 -- 2016}}
	\begin{itemize*}
		\item Maintenance, development and bug fixing ZF’s Middleware Library for various transmission ECU (TCU) projects (C).
		\item Requirements analysis and implementation from both the customer (ZF) and OEMs (e.g. BMW).
		\item AUTOSAR communication stack configuration and adaptation (CAN and FlexRay).
		\item UDS (ISO 14229-1- 2013) diagnostic routines and DIDs configuration. Analyse UDS implementation (NRC handling for all services).
		\item Implementing system tests (python).
	\end{itemize*}

	\item
	\headerrow
		{\textbf{Visteon Electronics GmbH}}
		{\textbf{Monheim am Rhein, Germany}}
	\\
	\headerrow
		{\emph{Consultant - Senior Software Developer}}
		{\emph{2014 -- 2014}}
	\begin{itemize*}
		\item Bug-fixing Autosar SWCs for the VW instrument cluster, MQB platform (C).
		\item Implementing costumer requirements (C).
	\end{itemize*}


	\item
	\headerrow
		{\textbf{Bosch Car Multimedia GmbH}}
		{\textbf{Hildesheim, Germany}}
	\\
	\headerrow
		{\emph{Consultant - Senior Software Developer}}
		{\emph{2014 -- 2014}}
	\begin{itemize*}
		\item Bug-fixing Autosar SWCs for the Gen3 car radio (C).
		\item Implemented reset and wakeup sources monitoring (V850) (C).
		\item Defined the concept and implemented the exception handling for V850.
	\end{itemize*}


	\item
	\headerrow
		{\textbf{3Soft Automotive SRL}}
		{\textbf{Sibiu, Romania}}
	\\
	\headerrow
		{\emph{Senior Software Developer}}
		{\emph{2012 -- 2014}}
	\begin{itemize*}
		\item Designed, developed and coordinated the AUTOSAR Xcp slave module (C).
		\item Performed integration tests with CANape.
		\item Implemented a plugin for an Eclipse based AUTOSAR BSW configurator to generate
		ORTI files based on the system description (Java).
		\item Offered customer support with Xcp technical issues.
	\end{itemize*}

	\item
	\headerrow
		{\textbf{Continental Automotive Systems SRL }}
		{\textbf{Sibiu, Romania}}
	\\
	\headerrow
		{\emph{Summer Intern, Junior Software Developer, Software Developer}}
		{\emph{2007 -- 2012}}
	\begin{itemize*}
		\item Implemented an automatic test framework (\Csharp).
		\item Maintained and integrated the AUTOSAR memory stack into the legacy ECU SW (C);		.
		\item AUTOSAR MCAL development for TI and Freescale uC (C).
		\item Maintenance, configuration management, debugging, testing and release management of existing embedded application (Electronic Brake System) - mostly for hardware related software
		\item Planned and performed system tests (HIL and SIL).
		\item Trainer for Autosar Memory Stack and V-Cycle development.
		\item Implemented a wizard to configure the ECU’s entire NV memory stack starting from the legacy configuration end ending up with the AUTSAR memory stack (Java).
		\item Implemented a tool to fetch MKS (now known as PTC) requirements from the server, translate them, and upload them back (Java).
		\item Implemented a prototype for employee’s training database (Javascript).
	\end{itemize*}

\end{itemize}


\hrule
\vspace{-0.4em}
\subsection*{Education}

\begin{itemize}
	\parskip=0.1em

	\item 
	\headerrow
		{\textbf{Lucian Blaga State University}}
		{\textbf{Sibiu, Romania}}
	\\
	\headerrow
		{\emph{Faculty of Engineering, B.S. Computer Science}}
		{\emph{2003 -- 2008}}

\end{itemize}


\hrule
\vspace{-0.4em}
\subsection*{Core Technical Skills}

\begin{indentsection}{\parindent}
\hyphenpenalty=1000
\begin{description*}
	\item[Languages:]
	C (expert), \CPP (proficient), Java (prior experience), \Csharp (prior experience), python (prior experience)
	\item[Methods:]
	Agile, Design Patterns, Clean Code, Clean Architecture, SOLID principles, OCI containers
	\item[Tools:]
	Canape, Canoe, git, PTC, Doors, WinIdea Genivi Dlt Viewer, Tresos, DaVinci, Eclipse, Visual Studio Code, too many to mention :) 
\end{description*}
\end{indentsection}

\end{document}
